\documentclass[8pt]{beamer}
\setbeamertemplate{navigation symbols}{}  % remove navigation symbols

\usepackage[utf8]{inputenc}
\usepackage{listings}
\usepackage{subcaption}
\usepackage{tikz}
\usepackage{amssymb,amsthm,amsmath}

\lstdefinelanguage{Julia}%
  {morekeywords={abstract,begin,break,case,catch,const,continue,do,else,elseif,%
      end,export,false,for,function,immutable,import,importall,if,in,%
      macro,module,otherwise,quote,return,struct,switch,true,try,typealias,%
      using,while,Union},%
   sensitive=true,%
   alsoother={\$},%
   morecomment=[l]\#,%
   morecomment=[n]{\#=}{=\#},%
   morestring=[s]{"}{"},%
   morestring=[m]{'}{'},%
}[keywords,comments,strings]%


\lstset{
    language = Julia,
    escapeinside={/*}{*/},
    basicstyle=\normalfont
}

\lstdefinelanguage{sml}{}

% code inline
\newrobustcmd{\sml}[2][]{{\sloppy
\ifmmode
\text{\lstinline[language=sml,#1]`#2`}
\else{\lstinline[language=sml,#1]`#2`}%
\fi}}

\newrobustcmd{\julia}[2][]{{\sloppy
\ifmmode
\text{\lstinline[language=Julia,#1]`#2`}
\else{\lstinline[language=Julia,#1]`#2`}%
\fi}}

\newcommand{\yikes}{\color{red}\textbf{ (!)}}

\begin{document}

\begin{frame}[fragile]
  \frametitle{Type Unions}

  Terminology: \emph{types} are prescriptive. Some examples:
  \begin{align*}
    \sml{unit}  &:= \sml{()} \\
    \sml{bool}  &:= \sml{false} \mid \sml{true} \\
    \sml{order} &:= \sml{LESS} \mid \sml{EQUAL} \mid \sml{GREATER} \\
    \sml{nat}   &:= \sml{zero} \mid \sml{succ(nat)}
  \end{align*}

  \textit{Product types} (Cartesian product, tuples): $\tau_1 \times \tau_2$.

  \noindent\rule{\linewidth}{0.4pt}

  What about ``combining'' two types? One idea: $\julia{Union\{}\tau_1,\tau_2\julia{\}}$. However:
  \begin{align*}
    &\julia{const MightFail\{T\} = Union\{T,String\}} \\
    &\julia{const IntMightFail = MightFail\{Int\}} &&\julia{Union\{Int,String\}} \\
    &\julia{const StringMightFail = MightFail\{String\}} &&\julia{String} \yikes
  \end{align*}

  Alternative: \textit{sum types} (``tagged disjoint unions''): $\tau_1 + \tau_2$.
  \begin{align*}
    &\julia{const MightFail\{T\} = Ok\{T\} | Error\{String\}}
  \end{align*}

  Benefits:
  \begin{itemize}
    \item More modular -- cases are necessarily disjoint.
    \item Easier to develop efficient data structures, since sums are ``concrete''.
    \item Guaranteed \emph{case exhaustivity} checking! Easy to refactor code.
  \end{itemize}
\end{frame}

\end{document}
